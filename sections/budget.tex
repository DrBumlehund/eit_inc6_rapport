In the business plan, there is a need for a budget. Because it helps getting knowledge about cash flow and when the project starts earning money. There are different kinds of budgets, e.g. establishing budget, operating budget, cash flow budget.


\subsection{Establishing Budget}
The establishing budget shows how much money is needed, before the product is ready to get out to the customers, e.g. rent and wages.


\begin{table}[H]
\centering
\caption{The Establishing Budget for Get Hooked}
\label{tab:bud:estab}
% subfile with establishing budget table:
\subfile{tables/establishingBudgetTable.tex}
\end{table}

The establishing budget, lists the costs that the project has, before the product is ready for launching. As seen in \autoref{tab:bud:estab} The cost for the development period is estimated to be around 210.044 DKK. This covers necessary resources and used hours for development of the platform. There has been reserved a little amount of money each month to counsellors (lawyers and advisors, etc.), if needed.

\subsection{Operating Budget}

The operating budget shows the expected increase in customers, and earned money over a period. It can also be used to indicate, when the project can be expected to have a surplus and when the establishing budget is earned.

\begin{table*}[t!]
\centering
\caption{Operating Budget on a five year Schedule}
\label{tab:bud:oper}
% subfile with Operating budget table:
\subfile{tables/operating_budget.tex}
\textit{See a Operating budget for the first year in appendices \ref{tab:bud:oper:fy}}
\end{table*}


\autoref{tab:bud:oper} displays a five years estimation of the project budget. As seen in the budget, the total deficit is -414.830 DKK. at its peak, 8 months in. The deficit will last 15 months, after which the bank balance will become positive. It is important to note that all of this is built on the assumption, that the customer base continues to grow. An exponential  growth is expected, which means that the growth is predicted to start slowly, but pick up in speed, as the geographical reach is increased.

The budget, estimates 40.000 DKK per month, to pay the staff wages throughout the first year. After the first year, every staff member will get a wage of 30.000 DKK each month. There is estimated a growth in staff every year.

After a year, it is estimated that there is a need for an office. The office’s monthly rent is 14.250 DKK (based on an ad in Odense). Furthermore there is an additional 1.750 DKK every month, for other fixed costs. 

The operating budget estimates a turnover of 7.12 million DKK over five years. 


\subsection{Cash Flow Budget}
Cash flow budget shows the flow of money in and out of the banks.

\begin{table*}[t!]
\centering
\caption{The Cash Flow Budget}
\label{tab:bud:cflow}
% subfile with cash flow budget table:
\subfile{tables/cashFlowBudgetTable.tex}

\end{table*}

The cash flow budget shows the constant flow of money. As seen in \autoref{tab:bud:cflow} the project starts with 625,000 DKK, which is the total amount expected to be needed before the project starts making money. Which is the deficit for 15 month and the establishment budget. In the Cash flow budget every 3rd month there is the payment of VAT. Every month there is also taxes to pay, which is 22\%, when the money continues to stay in the company bank account. After 12 months there will be a total of 280,458 DKK in the company bank account. The start up money either origins from a bank loan or an investor. The operating budget, is made with the expectation that it will be a bank loan and therefore, there is a monthly installment on the bank loan, which is estimated to 5,800 DKK.


\subsection{Investors Profit}
In case that the project gains an investor, who wish to earn a certain amount of money before (s)he will invest in it. Therefore the group expects to sell 15\% of the project for 625,000 DKK. This means that the total project is valued to 4,167,000 DKK. 

