\subsection{Association Technique}

\begin{table*}[t!]
\centering
\caption{Idea Assessment Table}
\label{tab:iass}
\begin{tabular}{|l|r|l|l|l|l|}
\hline
\multicolumn{2}{|l|}{\textbf{Idea Assessment Table. Point given 1-5, 5 is highest score}} & \multicolumn{4}{l|}{\textit{Idea number}} \\ \hline
Requirements: & \multicolumn{1}{l|}{\textit{Weights:}} & T:1 & E:3 & B:4 & C:6 \\ \hline
Can we earn money? & \textit{40 \%} & 13 & 9 & 7 & 5 \\ \hline
Does it give local knowledge to foreign people in new places. & \textit{15 \%} & 10 & 12 & 19 & 4 \\ \hline
Will people use it? & \textit{15 \%} & 14 & 9 & 14 & 10 \\ \hline
is it possible to create? & \textit{30 \%} & 20 & 11 & 19 & 20 \\ \hline
Is it sustainable? & \textit{20 \%} & 15 & 8 & 16 & 8 \\ \hline
\multicolumn{2}{|r|}{\textit{Total, with weights:}} & \cellcolor[HTML]{34FF34}17.7 & \cellcolor[HTML]{FE0000}8.05 & \cellcolor[HTML]{F56B00}16.65 & \cellcolor[HTML]{FE0000}11.7 \\ \hline
\end{tabular}
\end{table*}

\subsubsection{Association Chain}
The associations chain initiates with one group member saying a random word, then the next member says the first word, they think of, when they hear the word. Thereafter the third member of the group says the first word they think of. This continues until there is about 20 words in the chain. After the chain is finished, the next part is the Focus part.

\subsubsection{Focus}
The focus question, is the same as the problem formulation namely; “How to obtain local knowledge, when in a foreign location”. The focus is important because it has a big influence in which direction the ideation is going. 

\subsubsection{Ideation}
Ideation is where the group discus every word, one by one and try to find a solution to the focus part, by using the words from the association chain. Here it is important to not set limitations to the answers, because it can limit the ideas, that will come out of this part.

\subsubsection{Elaboration}
At the elaboration we need to clarify our ideas from ideation. The best way to do that is to use sentences like: how can it be done, is it possible and how can we make it happen.

Ideas chosen with the Idea Sieve:
\begin{enumerate}
	\item List over places to rent fishing-equipment. 
	\item Eco friendly coloring to put in the ocean, to show the current, should disappear fast.
	\item A platform, that shows summer fitness possibilities, near you.
	\item A platform where people write down their sports and when they can meet.
	\item Platform to show where to buy/borrow a new EPI-pen.
	\item Platform for sharing leftover cake/food      
\end{enumerate}

Out of the six ideas,  there have been chosen four ideas to work with in the circle technique.

\subsubsection{Circle Technique}
The circle technique has three steps, the first is a short description of the ideas. When that is done the description is send to the next group member. In step two the group member is trying to make improvements to the idea . When every member has been looking for improvements, you move on to the final step. In the third and final step, all the improvements will be evaluated. The improvements, that the group find best are combined with the original idea (Appendix with the four ideas). After the ideas has been through the circle technique, the idea assessment phase is initiated.

See the ideas created in the circle technique in appendix \ref{app:circ}.

\subsubsection{Idea Assesment}
As seen in \autoref{tab:iass}, the ideas was rated in a rating system to assess our idea. First of all we decided to choose five requirements. These requirements have to be something that describes a good idea to work further with. The requirements are not equally weighted, therefore we gave them a percentage to see how important they are to us. Idea number one got the highest score, that's why we decided to change direction a bit from our first thoughts about this project, because we have realised that idea number one would be the best project for us. 


